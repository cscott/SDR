% calldefs.tex
% documentation for SDR's call definition format.

\documentclass[12pt]{article}
\usepackage{tex-squares}%\setlength{\parindent}{1cm}
\usepackage{parskip}
\usepackage{url}
\usepackage{listings}
\title{The SDR call definition language}
\author{C. Scott Ananian}
\date{}

% define our own language for the listings package
\lstdefinelanguage{CallFile}
  {morekeywords=[1]{def:,in:,from:,call:,optional:,prim:,xpart:,part:,ipart:,spoken:,condition:},%
   morekeywords=[2]{LEFT,ANY,FACING,DANCERS,RH,MINIWAVE},%
   sensitive=true,%
   morecomment=[l]//,%
   morecomment=[s]{/*}{*/},%
   alsoletter={:},%
   alsodigit={/},%
  }[keywords,comments]%
\lstset{language=CallFile,columns=flexible,%
   identifierstyle=\ttfamily,keywordstyle=[2]\sffamily,%
   xleftmargin=20pt,xrightmargin=10pt,%
   breaklines=true,%
   breakatwhitespace=true,prebreak=\textbackslash,%
   escapechar=\#,%
   captionpos=b,abovecaptionskip=12pt
}

\newcommand{\clause}[1]{\texttt{#1}}
\renewcommand{\call}[1]{\texttt{#1}} % defined in the tex-squares package
\newcommand{\package}[1]{\url{#1}}
\begin{document}
\maketitle

\section{Introductory examples}

Although a few primitives and concepts are defined procedurally in the
SDR code base (see the \package{net.cscott.sdr.calls.lists}
package), the majority of the call and concept definitions in SDR are
defined using a simple human-readable language in resources stored
in \texttt{*.call} resource files in \package{net.cscott.sdr.calls.lists}.

\subsection{Calls defined in terms of others}
Here is a simple example call definition:
\begin{lstlisting}
def: dosado to a wave
  in: 6
    call: dosado
    call: step to a wave
\end{lstlisting}

This defines a call named ``dosado to a wave'' which completes in 6
beats of music (according to the standard Callerlab timing documents)
and consists of two parts (which are themselves calls): \call{dosado}
followed by \call{step to a wave}.\footnote{If you've read Bill
  Ackerman's ``Miscellaneous C4 Notions'' book
  (\url{http://lynette.org/sd/book3.pdf}), you might be objecting that
  \call{dosado} is fractionalizable, but not actually divisible into
  parts.  Dancers would likely object to \call{initially tandem dosado
    to a wave}, for example.  Rest assured that the grammar can
  express this distinction; this example is slightly simplified.}
Indentation is significant, as are new lines.\footnote{Long lines can
  be split by using a backslash to escape the newline.}
Keywords (such as \clause{def}, \clause{in} and \clause{call}) are
typically followed by a colon, but they are reserved only in certain
contexts.\footnote{There is no problem defining or using the call
  \call{pass in} for example, even though ``in'' is a keyword, used both
  to describe timing and as part of the syntax of the \clause{prim}  clause.}

Here are two more simple definitions:
\begin{lstlisting}
def: couples circulate
  call: as couples(circulate)

def: couples trade
  in: 6
  call: as couples(trade)
\end{lstlisting}

You'll notice that the \clause{call} clauses contain what looks like
function applications: the concept \call{as couples} is applied to
the call \call{circulate} or \call{trade}, respectively.  The
\call{couples trade} call explicitly specifies the number of beats of
the resulting call, but this is optional: \call{couples circulate}
uses the default timing.  In the case of the \call{as couples}
concept, the timing defaults to the timing of the base call, so
\call{couples circulate} takes 4 beats, just like \call{circulate}.

\subsection{Calls defined using primitives}
Let's look at a more complicated example to see the primitives that
ultimately make up a call definition.  Listing~\ref{lst:passthru}
shows the definition of the call \call{pass thru}.

\begin{lstlisting}[numbers=left,float,caption=A definition with primitives.,label=lst:passthru]
def: pass thru
  optional: LEFT
  in: 2
  // define the common case first: we've got two parts there
  from: FACING DANCERS
    prim: 0, 1, none
    prim: 0, 1, none
  // from mixed formation, this is indivisible
  from: mixed(FACING DANCERS, RH MINIWAVE)
    ipart:
      from: FACING DANCERS
        prim: 0, 1, none
        prim: 0, 1, none
      from: RH MINIWAVE
        call: step thru // "rear back and..."
\end{lstlisting}

In the first line we see that this is a definition of a call named
``pass thru''.  The \clause{optional} clause in the second line
contains a set of comma-separated tags which are used as hooks to
provide extra rules for fitting this definition into the call grammar.
In this case, the \texttt{LEFT} tag signifies that we will
automatically create an additional call \call{left pass thru} with
the definition \texttt{left(pass thru)}, where \call{left} is a
concept very similar (but not identical) to the C3B concept
\call{mirror}.  The third line says that this call will take 2 beats.

The \clause{from} clauses in lines 5 and 9 introduce alternative
definitions.  Each \clause{from} clause will be tried, in sequence,
until one works.  The argument to \clause{from} is usually a formation
name.  To be specific, the argument specifies a \texttt{Matcher}
expression, which takes a \texttt{Formation} and returns a
\texttt{FormationMatch} object.  For example, the \texttt{MINIWAVE}
matcher is equivalent to the matcher expression \texttt{OR(RH\_MINIWAVE,
  LH\_MINIWAVE)}; that is, it matches either a left-handed or a
right-handed miniwave, setting dancer tags such as \texttt{BEAU} and
\texttt{BELLE} appropriately.  Available \texttt{Matcher}s are defined
in the \package{net.cscott.sdr.calls.MatcherList} class in SDR.

\begin{figure}
\displaytwo
{ \dancer ~s \cr \dancer ~n }%
{\texttt{MatcherList.FACING\_DANCERS}}%
{ \dancer ~n & \dancer ~s }%
{\texttt{MatcherList.RH\_MINIWAVE}}
\caption{The patterns matched by the \textsf{FACING DANCERS} and
  \textsf{RH MINIWAVE} expressions.}
\label{fig:facingdancer}
\end{figure}

The first alternative, lines 5 through 7, applies when the formation
matches \texttt{FACING DANCERS}; that is, when every dancer in the
formation can be assigned to an instance of the \texttt{FACING
  DANCERS} pattern (see figure~\ref{fig:facingdancer}).  We specify
two primitive movements, to allow the call to be fractionalized into
halves.  The arguments to prim are, in order, ``sashay movement'' (positive is
sashay to the right, negative is sashay to the left), ``forward
movement'' (positive is forward, negative is backward), and
``turning'' (right, left, in, out, none).\footnote{In order for
  turning direction to be unambiguous, the turn amount is never
  allowed to exceed a quarter turn in any one prim.}
Dancers are nominally 2 steps apart after breathing in both x
and y axes.  Line 6 thus moves the dancer halfway across by moving one
step forward, and line 7 completes the motion.\footnote{Note that ``do
  one half of a pass thru'' from facing dancers would result in a
  collision, which is resolved to right hands by the SDR engine as part of the
  breathing process.}

\begin{figure}
\displayone
{            &            & \dancer 3s \cr
  \dancer 1n & \dancer 2s & \cr
             &            & \dancer 4n }{}
\caption{A mixed collection of facing dancers and miniwaves.  The call
  ``pass thru'' is valid from this formation.}
\label{fig:passthru}
\end{figure}
Line 9 introduces another alternative definition, which handles the
Callerlab ``Ocean Wave Rule''.  The \clause{from} expression is more
complex here: it uses the \texttt{mixed()} matcher combinator to match
a mixed collection of facing dancers and miniwaves.  Line 10 states that this
definition is not fractionalizable or divisible into parts, since it's
not clear what \call{do half of a pass thru} would mean for the
formation in figure~\ref{fig:passthru}.

Lines 11 through 13 deal with the \textsf{FACING DANCERS}
subformations in the mixed collection, and they give the same
definition as lines 5 through 7.   Line 14 says that from dancers in a
right-hand miniwave, pass thru is the same as step thru.

\subsection{Calls with arguments}

Before we deal with the language more rigorously, let's walk through
a much more complicated example.  Listing~\ref{lst:squarethru} shows
the definition of the call \call{square thru}.\footnote{Again, the listing is
  slightly simplified, since the actual definition adds a bit of
  complexity to ensure that \call{square thru $N$} has exactly
  $N$ parts.}

\begin{lstlisting}[numbers=left,float,caption=A definition with arguments.,label=lst:squarethru]
def: square thru(n=4)
  optional: LEFT
  spoken: square thru (<n=number> (hands (around|round)?)?)?
  from: ANY
    condition: and(greater([n],0),not(greater([n],1)))
    call: _fractional([n], _in(2, pull by))
  from: ANY
    condition: greater([n], 1)
    call: _in(2, _square thru part)
    call: left(square thru(_subtract num([n],1)))
\end{lstlisting}

This is a recursive definition.  Let's take it line by line.

The \clause{def} clause says that this is a definition of a call named
``square thru'' which takes a single parameter, which we will refer to
as $n$ in the body of the definition.  The parameter has a
default value, 4.  If no argument is supplied, $n$ will be set
to 4 when the definition is evaluated.

The \clause{optional} clause in the second line specifies that we will
automatically create an
additional call \call{left square thru(n)} with the definition
\call{left(square thru([n]))}, as described in the previous example.

The \clause{spoken} clause in line 3 provides a regular-expression style grammar
for a set of spoken words which are considered to refer to this
definition.  The \clause{spoken} clause is optional for calls with no
arguments, defaulting to the call name.\footnote{Calls whose names
  begin with an underscore are considered ``internal''; they do not
  have a default spoken representation.  You could flout
  convention by providing a \clause{spoken} clause for a call whose
  name begins with an underscore, although this is obviously not
  encouraged without good reason.}
The \clause{spoken} clause also names the grammar production used for
any arguments.  Calling ``square thru three'' (or writing ``square
thru 3'') will result in the function application \texttt{square
  thru(3)}.\footnote{``Function names'' can contain spaces, as shown
  here.  In general, identifiers consist of a \textit{sequence} of ``words''.}
Note that in this case ``square thru'' is an
acceptable spoken version of the call; the numeric argument takes its
default value of 4 since it is not matched in the grammar.  Other
valid spoken word sequences are ``square thru 2 1/2'' and ``square
thru three hands round''.

The \clause{from} clauses in lines 4 and 7 introduce alternatives, as
described in the previous section.  In this case, the \texttt{ANY}
matcher in line 4 matches any formation.  It is used here to introduce
one of two alternative definitions for the call without constraining
the formation.  Often alternative definitions are distinguished by the
formations from which they are applied, but in the case of ``square
thru'' we are going to use a \clause{condition} clause to distinguish
the definitions.

The \clause{condition} clause evaluates a boolean expression, halting
evaluation of this alternative\footnote{Evaluation is halted by
  throwing a \texttt{BadCallException}.} if the condition evalutes to
false.  Available boolean functions are defined in
\package{net.cscott.sdr.calls.PredicateList}.  Note that arguments to
boolean predicates can establish other, non-boolean, contexts.  In
this case the \call{greater} predicate establishes a numeric context for
the evaluation of \texttt{[n]} (which is a reference to the argument
named $n$), \texttt{0} and \texttt{1}.  A predicate such as
\texttt{formation(MINIWAVE)} could also establish \texttt{Matcher} or other
contexts for the evaluation of arguments.  In line 5, this particular
expression restricts this alternative to apply only in the base case
of the definition: ``square thru $n$'' for $0 < n \leq 1$.

Line 6 provides a \clause{call} clause providing the definition
applicable for this alternative: do the fraction $n$ of a
``pull by'' taking 2 beats.  That is, if $n=1/2$, then do half of a
pull by, taking 1 beat.  (``Pull by'' is the same as a ``right pull by.'')

Lines 7 through 10 introduce a new alternative, applicable when $n>1$
(as specified in line 8).
It begins with the internal call \call{\_square thru part}, done in 2
beats (line 9).  This will be given its definition (``pass thru and face in'')
elsewhere in the file.  After this part of the call is complete, on
line 10 we recursively
invoke the call ``left square thru $n-1$'', written as an application
of the \call{left} concept to a recursive invocation of \call{square thru}.

\section{Clause Reference}

\subsection{Definition and grammar clauses}
\subsubsection{def}
\subsubsection{optional}
\subsubsection{spoken}
\subsubsection{example}
% main part of definition is ``piece'' which is restriction, opt, par, seq

\subsection{Selector and alternative clauses}
\subsubsection{from}
\subsubsection{select}

\subsection{Timing and restriction clauses}
\subsubsection{in}
\subsubsection{condition, ends in}

\subsection{Sequence clauses}
\subsubsection{part, ipart, xpart, fpart}
\subsubsection{call}
\subsubsection{prim}


\section{Idioms}
There are a number of nonstandard internal calls which are used
frequently inside SDR definitions.

\subsection{\_quasi concentric}
The \call{\_quasi concentric} concept is like \call{concentric},
except that no ``lines to lines, columns to columns'' or ``opposite
elongation rule'' adjustment is done.  The elongation of the resulting
formation is exactly the same as the original formation.  This is
often used in calls which are informally defined with ``centers'' and
``ends''.   For example, from facing lines ``centers face left while the
ends face left'' ends in a right-hand column, while ``concentric face
left'' ends in a two-faced line due to the lines-to-lines rule.  These
two calls are expressed in SDR as \call{\_quasi concentric(face left,
  face left)} and \call{concentric(face left, face left)},
respectively.

\displayone
{ \dancer 1s & \dancer 2s & \dancer 3s & \dancer 4s \cr
  \dancer 8n & \dancer 7n & \dancer 6n & \dancer 5n }%
{Starting formation}
\displaytwo
{ \dancer 1e & \dancer 2e & \dancer 3e & \dancer 4e \cr
  \dancer 8w & \dancer 7w & \dancer 6w & \dancer 5w }%
{After ``centers face left while ends face left''}
{ \dancer 1e & \dancer 4e \cr
  \dancer 2e & \dancer 3e \cr
  \dancer 7w & \dancer 6w \cr
  \dancer 8w & \dancer 5w }%
{After ``concentric face left''}

\subsection{\_o concentric}
The \call{\_o concentric} is similar to \call{\_quasi concentric}, in
that it does no elongation adjustment.  It goes further, however, in
that the ends are not expected to mentally ``breathe together'' before
finding the formation for their call (as the ends would to find their
box when executing ``concentric scoot back'' from two-faced lines, for
example).  When an informal definition says ``centers cross back while
the ends O circulate twice''\footnote{This is ``like a settle back'',
  a C4 call.} we can't use the \call{concentric} or \call{\_quasi
  concentric} concepts, because the ends won't have the required O
formation after they've mentally breathed in.  The centers part is
still concentric: ``own the centers cross back by twice O circulate'' is not
what is intended.  ``Own the centers concentric cross back by twice O
circulate'' is the intended precisely-stated definition, but that
would require use to internally create phantom ends with the proper
facing directions to allow them to do their part of the concentric
cross back.  We use the simpler definition \call{\_o concentric(cross
  back, 2(o circulate))}.

\displayone
{ \dancer 1n & \dancer 2s & \dancer 3n & \dancer 4s \cr
  \dancer 8n & \dancer 7s & \dancer 6n & \dancer 5s }%
{Starting formation}
\displaytwo
{ \dancer 1n & \dancer 4s \cr
  \dancer 8n & \dancer 5s }%
{\ctablebox{Ends evaluate in this formation \\ for \call{concentric} or
  \call{\_quasi concentric}.}}
{ \dancer 1n & \pdancer ~x & \pdancer ~x & \dancer 4s \cr
  \dancer 8n & \pdancer ~x & \pdancer ~x & \dancer 5s }%
{\ctablebox{Ends evaluate in this formation \\ for
  \call{\_o concentric}.}}

\subsection{\_maybe touch}\label{sec:maybetouch}
The \call{\_maybe touch} pseudo-concept is used to wrap the first part
of a call where the Callerlab ``Facing Couples Rule'' applies.  This
ensures that ``swing thru'' is valid from facing couples and from
mixed formations:
\displayone
{            &            & \dancer 3s & \dancer 5s \cr
  \dancer 1n & \dancer 2s &            &            & \dancer 7n & \dancer 8s\cr
             &            & \dancer 4n & \dancer 6n}%
{
Legal starting formation for \call{grand swing thru}.
}


Recall that one distinction between the call \call{swing and mix} and the
two calls \call{swing} and \call{mix} is that the facing couples rule
applies to \call{swing and mix} but it does not apply to the A2 call
\call{swing}, which must be called from a wave.  SDR uses the following
definition for \call{swing and mix} to express this distinction:
\begin{lstlisting}
def: swing and mix // 3 part call
  part:
    call: _in(2, _maybe touch(_wave swing))
  part: 2
    call: mix
\end{lstlisting}
The \call{\_maybe touch} helper is also used to implement Callerlab's
similar ``Ocean Wave Rule'' when defining \call{box the gnat}.

\subsection{The designator concept}
It is worth reading Dan Neumann's article \textit{Designators and Concepts}
(\url{http://lynette.org/dconcepts.html}) for context on this concept.
Designators are terms such as ``heads'', ``boys'', ``centers'', and
``leaders''.\footnote{SDR distinguishes between ``primitive'' or
  ``intrinsic'' tags which are immutable properties of the dancer
  (``head'', ``girl'') and tags which are properties of the formation
  (``center'', ``belle'').}

\end{document}
